\documentclass[a4paper, 11pt, french, oneside]{book}
\usepackage[utf8]{inputenc}
\usepackage[T1]{fontenc}
\usepackage[francais]{babel}
\usepackage{graphicx}
\usepackage{charter}
\usepackage{color}

\begin{document}
  
 \begin{titlepage}
  \begin{center}
   \includegraphics[width=12cm]{linux.jpg}%
   \hspace{60pt}%
   \vspace{50pt}%
   \LARGE
   \textbf{Résumé du cours de Linux en Latex}\\
   \color{red}
   \textbf{Classe X81}\\
   Ambroise - Anouar - Quentin\\
   \vspace{50pt}%
   \includegraphics[width=8cm]{latex.jpg}%
  \end{center}
 \end{titlepage}
      
   \tableofcontents
   \frontmatter
   \mainmatter
 \chapter{MODULE : Les outils essentiels}
  \section{\Large  Les 7 premières commandes}
   \begin{flushleft}
    • \textbf {\textit{\color{red}whoaim}} : Renvoie votre login-name actuel.\\
    • \textbf {\textit{\color{red}hostname}} : Renvoie le nom de la machine sur laquelle vous travaillez.\\
    • \textbf {\textit{\color{red}date}} : Renvoie la date actuelle.\\
    • \textbf {\textit{\color{red}uname}} : Renvoie des informations sur le système actuel.\\
    • \textbf{\textit{\color{red}passwd}} : Permet au user de changer son mot-de-passe et permet à     l’administrateur ou le root de changer le mot-de-passe d’un user.\\
    • \textbf{\textit{\color{red}touch}} : Permet la création d’un fichier vide ou la mise à jour de la date de modification d’un ficher existant.\\
    • \textbf{\textit{\color{red}last}} : Renvoie la liste des utilisateurs qui se sont récemment connectés au système\\
   \end{flushleft}      
  \section{\Large Obtenir de l’aide }
	\subsection{Avec l’attribut --help}
	 Pour obtenir une aide rapide sur une commande, on utilise l’attribut --help à la suite de la commande.
	\subsection{Avec la commande man} 
     {\color{red}Man} est une commande qui permet d’obtenir de l’aide quant à l’utilisation,
     la syntaxe et les attributs des autres commandes Linux.\\
     La commande man s’utilise avec la syntaxe suivante : man [Nø section] [nom de la commande recherchée.\\
     Il peut arriver que man ne soit pas à jours et ne vous renvoie rien ou des informations lacunaires: dans ce cas vous pouvez mettre à jours la base de donnée de man grâce à la commande mandb.
  \section{\Large  Comprendre les outils du « SHELL»}
   A) \textbf{Le « TAB Completion »} : le SHELL possède la capacité de compléter vos commandes si vous tapez sur « \textbf{TAB} » et que celle-ci ne souffre d’aucune ambiguïté. Si votre commande souffre d’ambiguïté, tapez 2 x sur «TAB » pour obtenir une liste réduite de commande.\\\\
   B) \textbf{History} : Le SHELL référence l’ensemble des commandes que vous utilisez dans la console et est capable de vous les restituer.\\
    History référence un fichier qui conserve une trace des commandes tapées et ce de manière persistance même après reboot.\\\\
   C) \textbf{Les redirections} $\Rightarrow$ il existe trois canaux principaux : \\
   
	1) {\color{red}STDIN} : c’est l’entrée standard (généralement le clavier)\\
	
	2) {\color{red}STDOUT} : c’est la sortie standard (généralement l’écran)\\
	
	3) {\color{red}STDERR} : c’est le canal d’erreur (généralement vers un fichier)\\
	
      Les redirections permettent de rediriger chaque canal selon nos besoins.\\

• Exemple : il est possible de diriger la sortie standard vers un fichier plutôt que vers l’écran.\\\\
D) \textbf{Les pipes « | »} :
Le pipe permet de rediriger la sortie d’une commande dans l’entrée d’une seconde afin que la deuxième commande effectue un traitement sur le résultat de la première.
\chapter{MODULE : Les outils essentiels pour la gestion des fichiers}
\section{\Large L’arborescence du système de fichier} 
 Cette structure peut sensiblement varier en fonction des distributions.\\
$\Rightarrow$ Mais un tronc commun est communément admis c’est le « FHS : file hierarchy
standard »\\
$\Rightarrow$ Chaque arborescence de fichier en Linux prend toujours naissance avec le
« root directory » ou « / »\\
$\Rightarrow$ Depuis le « / » l’arborescence se dessine autour de dossiers fondamentaux
pour le fonctionnement du système.\\
Ce système de fichier peut être héberger sur un seul device de stockage\\
$\Rightarrow$ HDD\\
$\Rightarrow$ SSD\\
$\Rightarrow$ Etc\\
• Cependant, il est courant et conseillé d’isoler certains dossiers sur des devices différents.\\
$\Rightarrow$ Exemple de dossiers couramment isolé sur un autre device de stockage :\\
 /home : parce que c’est un dossier souvent très volumineux\\
 /var : parce que c’est un dossier pouvant saturé le système puisqu’il héberge les fichiers de type « dynamique »\\
$\Rightarrow$ Pour pouvoir réaliser cette isolation Linux se repose sur le système de « MOUNT ».\\
$\Rightarrow$ Mount permet de connecter une partie du système de fichier à un stockage physique particulier de la machine.\\
\textbf{Le principe du mount est donc de connecter des parties du système de fichier à la représentation du système de stockage.}
\section{\Large Lister les fichiers avec ls} 
 
• Lister les fichiers en Linux est essentiel puisque nous travaillons principalement en ligne de commandes.\\
\textbf{ls -a} : renvoie la liste de tous les fichiers et des dossiers présent dans le
répertoire courant.\\
\textbf{ls -lrt} : renvoie la liste des fichiers et des dossiers classés en fonction du temps
de dernière modification
 
\section{\Large Utiliser les SHELL wildcarts} 
 
• Le SHELL Linux possède la capacité de globbing :\\
$\Rightarrow$ C’est à dire que le SHELL est capable d’interpréter des symboles de
remplacements dans les commandes.\\
$\Rightarrow$ * : remplace plusieurs caractère inconnus.\\
$\Rightarrow$ ? : remplace un caractère inconnu.\\
$\Rightarrow$ [a-9] : remplace un caractère par un des caractères du
« range » défini.
 
 
\section{\Large Copier un fichier avec la commande cp}  
 
• Pour copier un fichier ou un dossier d’un emplacement à l’autre dans l’arborescence de
fichiers, vous devez utiliser la commande :\\
$\Rightarrow$ Pour un fichier : cp [SOURCE] [DESTINATION]\\
$\Rightarrow$ Pour un dossier : cp -R [SOURCE] [DESTINATION]
 
\section{\Large Travailler avec les dossiers}

• La commande cd (change directory)\\
$\Rightarrow$ Elle permet de se déplacer dans le système. Le chemin peut être absolu ou relatif.\\
$\Rightarrow$ cd . permet de rester dans le répertoire courant.\\
$\Rightarrow$ cd .. permet de remonter dans le répertoire parent.\\\\
• La commande mkdir(make directory)$\Rightarrow$ permet de créer un dossier dans le système.\\

• La commande rmdir (remove directory) $\Rightarrow$ permet de supprimer un dossier dans le système.


\section{\Large Utiliser les chemins absolus et relatifs} 

• \textbf{Un chemin absolu} est un chemin qui commence à la racine du système de fichier.
 Dans notre cas cette racine est « / » aussi appelé « root ».\\

• \textbf{Un chemin relatif} est un chemin qui commence à la position actuelle dans le système de fichier.

\section{\Large Déplacer un fichier avec la commande mv}

• Techniquement il est possible de déplacer un fichier avec la commande cp mais celle-ci à le désavantage de conserver une version du fichier à l’emplacement originel.\\
• Pour cela la commande mv a la capacité de recopier le fichier dans une autre partie de
l’arborescence de fichier tout en effaçant le fichier de son emplacement originel.\\
• Il est à noter que d’un point de vue système, renommer un fichier revient à déplacer(mv) ou copier se fichier(cp) avec un autre nom dans une [DESTINATION == SOURCE]

\section{\Large Supprimer un fichier ou un dossier avec la commande rm}

• La suppression de fichiers ou de dossiers en Linux se fait via la commande rm.\\
$\Rightarrow$ Pour supprimer un fichier : rm [SOURCE]\\
$\Rightarrow$ Pour supprimer un dossier : rm -r [SOURCE]\\
$\Rightarrow$ Pour supprimer un dossier sans confirmation : rm -rf [SOURCE]

\section{\Large Comprendre le Hard-Link et le Symbolic-Link} 

• Les systèmes Linux possèdent une caractéristique très utiles que l’on appel « Link »\\
• Il existe deux types de « Link »\\

$\Rightarrow$ Le Hard-Link : est un nom qui référence un « inode » qui lui même référence un bloc sur le périphérique de stockage.\\

$\Rightarrow$ Le symbolic-Link : est un nom qui référence un Hard-link

\begin{flushleft}
\includegraphics[scale=0.6]{link.jpg}
\end{flushleft}

\section{\Large Créer un link avec la commande ln}

• Pour créer un « Link » dans un système Linux, vous devez utiliser la commande ln\\
• Peut importe que vous vouliez créer un Hard-Link ou un Symbolic-Link\\
• ln [SOURCE] [LINK-NAME] : permet de créer un HARD-LINK\\
• ln -s [SOURCE] [LINK-NAME] : permet de créer un Symbolic-Link\\
 
 \section{\Large Trouver un fichier avec la commande find}
 
 • Pour configurer certains services ou simplement parfois pour exécuter un script, il est nécessaire de trouver le fichier de configuration ou le fichier script dans le système de fichier.\\
• Cela peut s’avérer difficile étant donné la quantité de fichier contenue dans un OS.\\
$\Rightarrow$ find [START-POINT] -name « [FILENAME] » : permet de chercher un fichier/dossier de manière récursive grâce à son nom à partir du point de départ.\\
$\Rightarrow$ find [START-POINT] -user « [USERNAME] » : permet de chercher tous les fichiers/dossiers appartenant à un utilisateur de manière
récursive grâce au nom de l’utilisateur à partir du point de départ.
 
\begin{flushleft}
\includegraphics[scale=0.5]{Find.jpg}
\end{flushleft}
 
\chapter{MODULE : Travailler avec des fichiers textes}
\section{\Large Comprendre VIM} 

• Par tradition, il existe deux logiciels d’édition couramments utilisés pour réussir à éditer des fichiers en ligne de commande afin de ne pas devoir sortir du « SHELL »\\
 
$\Rightarrow$ VIM (celui que nous allons voir)\\
$\Rightarrow$ Emacs ( le plus compliqué des deux à maitriser)\\

• Pour éditer un fichier texte avec « VIM » vous devez utiliser la commande vim [SOURCE DU FICHIER]\\

• vim fonctionne sous 3 modes distinct :\\
1) Le mode « command » : permet de sauvegarder, quitter, rechercher …\\
2) Le mode « insert » : permet d’éditer le texte ...\\
3) Le mode « visual » : permet d’effectuer des sélections dans le texte ...\\

• Il est très important de comprendre que chaque action ne peut se faire que dans le
mode qui lui est dédié

• En fonction des modes « VIM » possède plusieurs options :\\

- \underline{En mode COMMANDE :}\\
- :w (sauvegarde votre fichier)\\
- :q (quitte le fichier)\\
- :! ( force l’action)\\
- :wq ! (sauvegarde et quitte en forçant l’action)\\
- u (undo)\\\\
- \underline{En mode INSERT :}\\
- Vous pouvez simplement l’utiliser comme un éditeur de texte sans l’option copier – coller.\\\\
- \underline{En mode VISUAL :}\\
- D (delete la sélection)\\
- Y (copie la sélection)\\
- P (colle la sélection)\\

\section{\Large Travailler avec le pager less} 

• less [SOURCE] est une commande qui à la capacité d’organiser le texte en page pour le « SHELL ». \\

\section{\Large Lire un fichier texte grâce à cat et tac} 

• Certain fichiers texte sont suffisamment court pour ne pas requérir à less.\\
• La commande cat [SOURCE] est alors utile pour présenter le contenu du fichier dans le « SHELL »\\

\section{\Large Lire le début ou la fin d’un fichier avec head tail} 

• head -n[nombre de lignes] [SOURCE] : présente les 10 premières lignes du fichier dans le
« SHELL »\\
• tail -n[nombre de lignes] [SOURCE] : présente les 10 dernières lignes du fichier dans le
« SHELL »

\section{\Large Travailler avec la commande grep en dehors de |} 

• grep est une des commandes les plus utiles en Linux.\\
• Exemple : vous chercher tout les fichiers ou il est écrit « dhcp » dans votre système.\\
• grep -iR [Expression recherchée] 2>/dev/null\\
• -i : attribut qui rend grep case insensitive\\
• -R : attribut qui permet à grep de travailler de manière récursive\\
• 2>/dev/null : redirection des erreurs dans le /dev/null\\

\section{\Large Comprendre les « REGEXP »} 

• Les expressions régulière sont des modelé de texte utilisable
par des outils présent dans Linux comme grep\\
• Il ne faut pas confondre les « REGEXP » et le « globbing » Linux\\
$\Rightarrow$ Le « globbing » est interne au « SHELL »\\
$\Rightarrow$ Les « REGEXP » est générale et utilisable par toute les
commandes qui traite des chaines de caractère.\\

\begin{flushleft}
\includegraphics[scale=0.4]{REGEXP.jpg}
\end{flushleft}


\section{\Large Utiliser les commandes awk, sort, tr} 
• awk est une commande qui permet de découper un texte en fonction de ses délimiteurs.
\begin{flushleft}
\includegraphics{AWK1.jpg}
\end{flushleft}
\begin{flushleft}
\includegraphics{AWK2.jpg}
\end{flushleft}

• sort est une commande qui permet de trier le texte dans l’ordre alphabétique ou numérique.\\
$\Rightarrow$ sort [source] (alphabétique)\\
$\Rightarrow$ sort -n [source] (numérique)\\

• tr est une commande qui permet réaliser une traduction de certains caractères du texte en d’autres\\
$\Rightarrow$ tr [caractère(s) d’origine] [caractère(s) de remplacement]\\
$\Rightarrow$ tr [état d’origine] [état de remplacement]




 
   \appendix
   \chapter{Commandes utiles}
      \section{Section d'annexe}
    
   \backmatter
   
   \chapter{Épilogue}
\end{document}
