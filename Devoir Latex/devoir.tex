\documentclass[a4paper, 12pt, french]{book}
\usepackage[utf8]{inputenc}
\usepackage[T1]{fontenc}
\usepackage{babel}

\title{\textbf{{\LARGE Résumé du cours de Linux en Latex}}}
\author{QM - AM - AN}

\begin{document}
   \maketitle
   \tableofcontents
   \frontmatter
   \mainmatter
   \chapter{MODULE 1}
      \section{\Large  Les 7 premières commandes}
      
      
     \begin{flushleft}
     • \textbf {\textit{whoaim}} : Renvoie votre login-name actuel.\\
     • \textbf {\textit{hostname}} : Renvoie le nom de la machine sur laquelle vous travaillez.\\
     • \textbf {\textit{date}} : Renvoie la date actuelle.\\
     • \textbf {\textit{uname}} : Renvoie des informations sur le système actuel.\\
     • \textbf{\textit{passwd}} : Permet au user de changer son mot-de-passe et permet à l’administrateur ou le root de changer le mot-de-passe d’un user.\\
     • \textbf{\textit{touch}} : Permet la création d’un fichier vide ou la mise à jour de la date de modification d’un ficher existant.\\
     • \textbf{\textit{last}} : Renvoie la liste des utilisateurs qui se sont récemment connectés au système\\
     \end{flushleft}
     
       
\section{\Large Obtenir de l’aide }
		\subsection{Avec l’attribut "--help"}
	Pour obtenir une aide rapide sur une commande, on utilise l’attribut "- -help" à la suite de la commande.
		\subsection{Avec la commande man} 
 Man est une commande qui permet d’obtenir de l’aide quant à l’utilisation,
la syntaxe et les attributs des autres commandes Linux.\\
La commande man s’utilise avec la syntaxe suivante : man [Nø section] [nom de la commande recherchée.\\
Il peut arriver que man ne soit pas à jours et ne vous renvoie rien ou des
informations lacunaires: dans ce cas vous pouvez mettre à jours la base de donnée de man
grâce à la commande mandb.
\section{\Large  Comprendre les outils du « SHELL»}
A) \textbf{Le « TAB Completion »} : le SHELL possède la capacité de compléter vos
commandes si vous tapez sur « \textbf{TAB} » et que celle-ci ne souffre d’aucune
ambiguïté. Si votre commande souffre d’ambiguïté, tapez 2 x sur «TAB » pour
obtenir une liste réduite de commande.\\\\
B) \textbf{history} : Le SHELL référence l’ensemble des commandes que vous
utilisez dans la console et est capable de vous les restituer.\\
history référence un fichier qui conserve une trace des commandes
tapées et ce de manière persistance même après reboot.\\\\
C) Les redirections $\Rightarrow$ il existe trois canaux principaux : \\

	• STDIN : c’est l’entrée standard (généralement le clavier)\\
	
	• STDOUT : c’est la sortie standard (généralement l’écran)\\
	
	• STDERR : c’est le canal d’erreur (généralement vers un fichier)\\
	
Les redirections permettent de rediriger chaque canal selon nos besoins .\\

	• Exemple : il est possible de diriger la sortie standard vers un fichier plutôt que vers l’écran.\\\\
D) Les pipes « | » :
Le pipe permet de rediriger la sortie d’une commande dans l’entrée d’une seconde afin que la deuxième commande effectue un traitement sur le résultat de la première.
\chapter{MODULE 2}
\section{\Large L’arborescence du système de fichier} 
 Cette structure peut sensiblement varier en fonction des distributions
$\Rightarrow$ Mais un tronc commun est communément admis c’est le « FHS : file hierarchy
standard »\\
$\Rightarrow$ Chaque arborescence de fichier en Linux prend toujours naissance avec le
« root directory » ou « / »\\
$\Rightarrow$ Depuis le « / » l’arborescence se dessine autour de dossiers fondamentaux
pour le fonctionnement du système.\\
Ce système de fichier peut être héberger sur un seul device de stockage\\
$\Rightarrow$ HDD\\
$\Rightarrow$ SSD\\
$\Rightarrow$ Etc\\
• Cependant, il est courant et conseillé d’isoler certains dossiers sur des devices différents.\\
$\Rightarrow$ Exemple de dossiers couramment isolé sur un autre device de stockage :\\
 /home : parce que c’est un dossier souvent très volumineux\\
 /var : parce que c’est un dossier pouvant saturé le système puisqu’il héberge les fichiers de type « dynamique »\\
$\Rightarrow$ Pour pouvoir réaliser cette isolation Linux se repose sur le système de « MOUNT ».\\
$\Rightarrow$ Mount permet de connecter une partie du système de fichier à un stockage physique particulier de la machine.\\
\textbf{Le principe du mount est donc de connecter des parties du système de fichier à la représentation du système de stockage.}
\section{\Large Lister les fichiers avec ls} 
 
• Lister les fichiers en Linux est essentiel puisque nous travaillons principalement en ligne de commandes.\\
\textbf{ls -a} : renvoie la liste de tous les fichiers et des dossiers présent dans le
répertoire courant.\\
\textbf{ls -lrt} : renvoie la liste des fichiers et des dossiers classés en fonction du temps
de dernière modification
 
\section{\Large Utiliser les SHELL wildcarts} 
 
• Le SHELL Linux possède la capacité de globbing :\\
$\Rightarrow$ C’est à dire que le SHELL est capable d’interpréter des symboles de
remplacements dans les commandes.\\
$\Rightarrow$ * : remplace plusieurs caractère inconnus.\\
$\Rightarrow$ ? : remplace un caractère inconnu.\\
$\Rightarrow$ [a-9] : remplace un caractère par un des caractères du
« range » défini.
 
 
\section{\Large Copier un fichier avec la commande cp}  
 
• Pour copier un fichier ou un dossier d’un emplacement à l’autre dans l’arborescence de
fichiers, vous devez utiliser la commande :\\
$\Rightarrow$ Pour un fichier : cp [SOURCE] [DESTINATION]\\
$\Rightarrow$ Pour un dossier : cp -R [SOURCE] [DESTINATION]
 
\section{\Large Travailler avec les dossiers}

• La commande cd (change directory) $\Rightarrow$ Elle permet de se déplacer dans le système. Le chemin peut être absolu relatif.\\

• La commande rmdir (remove directory) $\Rightarrow$ Elle permet de supprimer un dossier dans le système.


\section{\Large Utiliser les chemins absolus et relatifs} 

• \textbf{Un chemin absolu} est un chemin qui commence à la racine du système de fichier.
 Dans notre cas cette racine est « / » aussi appelé « root ».\\

• \textbf{Un chemin relatif} est un chemin qui commence à la position actuelle dans le système de fichier.

\section{\Large Comprendre le Hard-Link et le Symbolic-Link} 

• Il existe deux types de « Link »\\

$\Rightarrow$ Le Hard-Link : est un nom qui référence un « inode » qui lui même référence un bloc sur le périphérique de stockage.\\

$\Rightarrow$ Le symbolic-Link : est un nom qui référence un Hard-link

\section{\Large Créer un link avec la commande ln} 


 
 
   \appendix
   \chapter{Annexe}
      \section{Section d'annexe}
    
   \backmatter
   \chapter{Épilogue}
\end{document}
